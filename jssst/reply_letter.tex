\documentclass{article}

\usepackage{a4wide}
\usepackage{graphicx}
\usepackage{fancybox}
\usepackage{ascmac}
\usepackage{xspace}
%\usepackage{pdfpages}
\usepackage{pifont}
\usepackage{array}

\newcommand\st{\textsuperscript{st}\xspace}
\newcommand\nd{\textsuperscript{nd}\xspace}
\newcommand\rd{\textsuperscript{rd}\xspace}

\begin{document}

% cover letter
\begin{flushleft}
  2017-01-13\newline 
  ソフトウェア論文特集\newline
  論文・記事番号:16-S-05\newline
  題目:Component-Based Framework of Lightweight Ruby for Efficient Embedded Software Development\newline
  著者: Takuro Yamamoto, Hiroshi Oyama, and Takuya Azumi\newline

\section{Response to 1\st reviewer}

\begin{enumerate}

\item \begin{flushleft}
    \textbf{Comment:
}初見では、Fig.21 をどう解釈すればよいかが分かりにくかったです。
図の解釈について説明を追加いただければ、多くの読者にとってわかりやすくなると思います。
  \end{flushleft}

  \begin{flushleft}
    \textbf{Our reply:}
Thank you for your suggestion.
We have added and modified the section as follows. (page 13)

\begin{itembox}[|]{4.3 Synchronization of Multiple RiteVMs}
To execute multiple mruby applications, a synchronization mechanism for RiteVM tasks is implemented in the proposed framework.
We measured the execution time from the execution of the first RiteVM task to that of the last RiteVM task.
Fig. 21 shows the case of two to eight RiteVM tasks.
Note that the periodic time is 1 msec.
It was confirmed that the execution time was within the theoretical value which is {\it\scriptsize periodic time $\times$ (number of RiteVM tasks - 1).}
In the case of two RiteVM tasks, the theoretical value is 1 msec and the execution time does not exceed the theoretical value. 
As shown in Fig. 21, the execution time is within the theoretical value, which indicates successful synchronization of multiple RiteVM tasks.
\end{itembox}\\
  \end{flushleft}


\item \begin{flushleft}
    \textbf{Comment:}
3ページ左カラムの下から2行目で「4. Benets of CBD:」で始まっていますが、他の項目は「.」で終わっています。
この点、ご確認ください。
  \end{flushleft}

  \begin{flushleft}
    \textbf{Our reply:}
Thank you for your careful reading.
We have modified other contributions to ``:'' such as "Improved software development efficiency:". (page 3)
  \end{flushleft}

\end{enumerate}

\section{Response to 2\nd reviewer}

\begin{enumerate}

\item \begin{flushleft}
    \textbf{Comment:}
Figure 11 seems to be a wrong diagram.
Maybe this is an old version.
Please replace it with a correct one so that A, B, C, D matches to the description in the text.
  \end{flushleft}

  \begin{flushleft}
    \textbf{Our reply:}
Thank you for your suggestion.
As you said, we used the different figure.
We have replaced it with the correct figure. (page 7)
  \end{flushleft}


\item \begin{flushleft}
    \textbf{Comment:}
The first reference of Figure 12 from the text is after Figure 13 is refereed.
Please change the order.
  \end{flushleft}

  \begin{flushleft}
    \textbf{Our reply:}
Thank you for your suggestion.
We have switched the order of Fig. 12 and Fig. 13. (page 7-8)
  \end{flushleft}


\end{enumerate}

\end{document}

