%卒業論文用雛形
\documentclass[a4j,12pt,oneside,openany]{jsbook}
% 英語なら以下を使う.
%\documentclass[a4j,12pt,oneside,openany,english]{jsbook}

\usepackage{graphicx}
\usepackage{amssymb}
\usepackage{amsmath}
\usepackage{latexsym}

%jsbook を report っぽくするスタイルファイル
\usepackage{book2report}
%定理,補題,系,例題,証明などや英語用の定義がされています.
%自分なりにいじってください.
\usepackage{thesis}
% 具体的には以下のように定義されています.
% 英語の定理環境
%  \newtheorem{theorem}{Theorem}[chapter]
%  \newtheorem{lemma}{Lemma}[chapter]
%  \newtheorem{proposition}{Proposition}[chapter]
%  \newtheorem{corollary}{Corollary}[chapter]
%  \newtheorem{definition}{Definition}[chapter]
%  \newtheorem{example}{Example}[chapter]
%  \newtheorem{proof}{Proof}
% 日本語の定理環境
%  \newtheorem{theorem}{定理}[chapter]
%  \newtheorem{lemma}{補題}[chapter]
%  \newtheorem{proposition}{命題}[chapter]
%  \newtheorem{corollary}{系}[chapter]
%  \newtheorem{definition}{定義}[chapter]
%  \newtheorem{example}{例}[chapter]
%  \newtheorem{proof}{証明}
% 証明には番号をつけず,最後は Box で終わります.

% 英語で,見出しのフォントが気に入らなかったら
%\renewcommand{\headfont}{\bfseries}

%ページ数が少ないときはここを大きくしてごまかそう!!効果絶大!!
\renewcommand{\baselinestretch}{1.1}

\begin{document}
%%%%%%%%%%%% 題目 %%%%%%%%%%%%%%%%%%%%%%%%%%%%%%%%%%%%%%%%%%%%%%%%%%%%%%%
%%%%%%%%%%%% ここも適当に変えてもいいと思う %%%%%%%%%%%%%%%%%%%%%%%%%%%%%%%%%
\thispagestyle{empty}
\begin{center}
\vspace*{5mm}
{\Huge {\bf 特 \hspace{12pt} 別 \hspace{12pt} 研 \hspace{12pt} 究 \hspace{12pt} 報 \hspace{12pt} 告}}\\
\vspace{2cm}
{\Large 題\hspace{8mm}目}\\
\vspace{1cm}
\underline{\LARGE{なんとかなんとかの}} \\
\vspace{0.5cm}
\underline{\LARGE{なんとかなんとか}} \\
\vspace{12mm}
{\large 指 導 教 員}\\
\vspace{6mm}
\underline{\Large なんとか だれそれ 教 授}\\
 \\
\underline{\Large なんとか だれそれ 准 教 授}\\
\vspace{8mm}
{\large 報 告 者}\\
\vspace{6mm}
\underline{\Large なんとか だれそれ}\\
\vspace{10mm}
{\Large 平成28年2月吉日}\\
\vspace{14mm}
{\Large 大阪大学基礎工学部システム科学科\\知能システム学コース}\\
\end{center}
\clearpage
\setcounter{page}{0}
\pagenumbering{roman}

%%%%%%%%%%%% 概要 %%%%%%%%%%%%%%%%%%%%%%%%%%%%%%%%%%%%%%%%%%%%%%%%%%%%
\begin{abstract}
おい おまえ! おれの名をいってみろ!!\\
そうか!! おまえおれの胸の傷を見ても誰かわからねぇのか?\\
そうか おまえ死にたいのか・・・\\
もう一度だけチャンスをやろう! おれの名をいってみろ!!\\
ほ〜〜〜 それではおれの名をいってみろ!!\\
おれはウソが大きれぇなんだ!! 不発か・・・ 運がよかったな ん!? おい\\
肝の小せぇやろうだ・・・ ショック死しやがった・・・\\
なにを勘違いしている この女はおれがもらった!!\\
フフフッ・・・おまえら〜〜〜〜 おれの名をいってみろ!!\\
\end{abstract}

%%%%%%%%%%%% 目次 %%%%%%%%%%%%%%%%%%%%%%%%%%%%%%%%%%%%%%%%%%%%%%%%%%%%
\clearpage
\tableofcontents
\clearpage
\setcounter{page}{0}
\pagenumbering{arabic}

%%%%%%%%%%%% 1章 %%%%%%%%%%%%%%%%%%%%%%%%%%%%%%%%%%%%%%%%%%%%%%%%%%%
\chapter{はじめに}
いきなり定理
\begin{theorem}[内田の定理]
内田が肯定した事実は全て真実である.そこに一切の虚偽,虚構は存在し得ない.
\end{theorem}
\section{はじめにのはじめに}
\section{はじめにのつぎに}

\begin{lemma}[最初の補題]
これは補題です.
\end{lemma}

\begin{proposition}[最初の命題]
これは命題です.
\end{proposition}

\begin{corollary}[最初の系]
これは系です.
\end{corollary}

\begin{proof}
これは自明です.
\end{proof}

\begin{definition}[最初の定義]
私はこれを「あれ」と呼びます.
\end{definition}

\begin{example}[最初の例題]
これは例題です.
\end{example}
%%%%%%%%%%%% 2章 %%%%%%%%%%%%%%%%%%%%%%%%%%%%%%%%%%%%%%%%%%%%%%%%%%%
\chapter{つぎに}

%%%%%%%%%%%% 3章 %%%%%%%%%%%%%%%%%%%%%%%%%%%%%%%%%%%%%%%%%%%%%%%%%%%
\chapter{なか押し}

%%%%%%%%%%%% 4章 %%%%%%%%%%%%%%%%%%%%%%%%%%%%%%%%%%%%%%%%%%%%%%%%%%%
\chapter{ダメ押し}

%%%%%%%%%%%% 5章 %%%%%%%%%%%%%%%%%%%%%%%%%%%%%%%%%%%%%%%%%%%%%%%%%%%
\chapter{むすび}

%%%%%%%%%%%% 謝辞 %%%%%%%%%%%%%%%%%%%%%%%%%%%%%%%%%%%%%%%%%%%
\begin{acknowledgement}
みんな本当にありがとう・・・.
\end{acknowledgement}

%%%%%%%%%%%% 参考文献 %%%%%%%%%%%%%%%%%%%%%%%%%%%%%%%%%%%%%%%%%%%%%%%%%%
%適当に変えてねー.
\begin{thebibliography}{99}
\bibitem{a} Reference 1
\bibitem{b} Reference 2
\end{thebibliography}
%\bibliographystyle{myjunsrt}
%\bibliography{refs}

\end{document}
